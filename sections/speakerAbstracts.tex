
\section{Lecture Abstracts}

\begin{center}
{{\large\bfseries Building Planets - A Journey along 40 Orders of Magnitude}\par} \medskip

{\large Til Brinstiel, Ludwig-Maximilians-Universität München\par}
\end{center}

\noindent Building planets is a dirty business. First of all, planets are made out of the dirt we call interstellar dust. Secondly, the physics involved is not "clean" in a sense that neither the processes involved, nor the initial conditions are known. Solid state physics, radiation transport, gas phase and surface chemistry, magnetic fields and hydrodynamic instabilities at high Reynolds numbers are just some of the aspects that are certainly involved in growing the sub-micrometer sized interstellar dust by 40 orders of magnitude in mass to a full-fledged planet. Given this complexity and dynamic range, it is perhaps not surprising, that the formation processes of planets are still poorly understood, even though thousands of planets beyond our solar system are known today.

Some of the biggest mysteries of planet formation lie in the early stages: growing the asteroid-sized building blocks of planets. Recent years have seen a revolution in observing capabilities at various wavelength ranges delivering data of unprecedented detail and sensitivity. They have partially confirmed our theoretical expectations, partially surprised us. In this lecture, I will discuss some of the basic concepts and the problems we are facing from the theoretical side. I will outline how they might be overcome and will show how recent observational break-throughs revoluzionize this exciting field, bringing us closer to solving the puzzle of planet formation.\par
\newpage


\begin{center}
{{\large\bfseries Galaxy Formation and Cosmology}\par} \medskip

{\large Tobias Buck, Max-Planck-Institut for Astronomy (MPIA), Heidelberg\par}
\end{center}

\noindent In these two lectures I will give a broad overview over the field of galaxy formation and evolution and the cosmological standard model.
Cosmology governs the evolution of the spacetime of our Universe. Thus, I will first discuss the reasoning and implications of the cosmological model and highlight our understanding about nucleosynthesis and structure formation within this model.
I will then move to describe the collection of observational facts about the galaxy population of our Universe in order to discuss the physical principles needed to explain these observations. Finally, I will briefly summarise current efforts and models in numerical galaxy formation to test physical mechanisms involved in the formation of galaxies.
 
\par
\newpage




\begin{center}
{{\large\bfseries Strongly Coupled Systems and the Applied Physics of Black Holes}\par} \medskip

{\large Carlo Ewerz, Institute for Theoretical Physics, Heidelberg University\par}
\end{center}

\noindent One of the most surprising findings of string theory is that quantum field theories and higher-dimensional gravity are closely connected, and can in fact be different descriptions of the same fundamental laws. The corresponding holographic dualities improve our understanding of fundamental theories of Nature and have many applications. In particular, they offer new and revolutionary insights into the behavior of strongly coupled quantum systems by connecting them to the physics of black holes. This has helped to discover universal behavior in many strongly coupled systems.
 
In the first part of the lecture, I will give a basic introduction to the principle of holographic duality. In the second part, I will present some examples and applications including the description of heavy quarks in a quark-gluon plasma and turbulence in a superfluid.
 
The lecture will start from basic (mostly undergraduate level) physics and does not require any expert knowledge of string theory etc.
\par
\newpage

\begin{center}
{{\large\bfseries Prospects and challenges of quantum computing}\par} \medskip

{\large Martin Gärttner, Kirchhoff-Institute for Physics, Heidelberg University\par}
\end{center}

\noindent “The first fusion reactor will be producing energy within 30 years from now.” This statement seems to be true independent of when it is made. Similarly, the Wikipedia page on quantum computing has been stating for quite a while now “the development of actual quantum computers is still in its infancy”. Will this statement ever be removed? The massive investments into quantum technologies that have been made by governments, companies, and military agencies especially over the last decade seem to suggest that the breakthrough is now within reach. 
In this lecture I will review some of the history of quantum computing and establish the basic language of quantum information consisting of quantum circuits built from unitary “quantum gates”. Armed with this toolbox we will explore a simple quantum algorithm, the Deutsch-Jozsa algorithm, which illustrates how quantum superposition makes quantum computers superior to their classical ancestors. The last part of the lecture is dedicated to challenges for quantum computing like unavoidable sources of decoherence and how to use quantum error correction to mitigate this problem.
\par
\newpage

\begin{center}
{{\large\bfseries Unlocking Changes in Ocean Dynamics}\par} \medskip

{\large Freya Hemsing, Institute for Environmental Physics, Heidelberg\par}
\end{center}

\noindent 
Our ability to resolve current and past variability of the Earth’s climate relies on our understanding of the circulation patterns and geochemical processes in the ocean. Interacting with the atmosphere, cryosphere and biosphere, the ocean occupies a key role in the global climate. The present oceanic system is measured directly in the scope of global scientific programs such as ARGO or GEOTRACES and simulated in computational models. Oceanic processes predominantly occur on time scales from a few decades to thousands of years. Therefore, to reliably constrain models for present and past oceanic changes, a look further back in the past, using geological records, is needed. Over the past decades the development of proxy-archive systems that record variations in oceanic conditions has been a key effort in paleoceanography. The most widely used archive is marine sediment, from which deep and surface ocean properties can be reconstructed. To detect changes in the thermocline (70 – 1000 m) which buffers and links the well mixed warm surface with the slow and cold deep water masses, the aragonite skeletons of cold-water corals are a suitable archive.

This lecture aims to provide a general overview of what is currently known about ocean dynamics and geochemical processes and how this knowledge was gained. A particular focus will be on the importance to understand the oceans’ role in Earth’s climate. In the second part, the benefits of paleoceanographic investigations will be outlined in form of a few example studies using both sediment and cold-water corals that yield insight into ocean properties and dynamics during the Last Glacial period.
\par
\newpage

\begin{center}
{{\large\bfseries Josephson junction based superconducting electronics}\par} \medskip

{\large Sebastian Kempf, Kirchhoff-Institute for Physics, Heidelberg University\par}
\end{center}

\noindent Advances in science, health care or other areas of everyday life are often accompanied by progress in physical instrumentation. The development of ultra-sensitive detectors and sensors is therefore of great importance and will not only influence our understanding of nature but also future examination methods in medical care or search strategies for natural resources.
Josephson junction based superconducting electronics devices play an important role for these developments as they are among the most sensitive measurement instruments presently existing that enable fascinating investigations of tiniest signals. Josephson junction based interferometers and amplifiers, for example, are very well suited for measuring variations of tiny magnetic fields or any other physical quantities that can be naturally converted into magnetic flux. They are based on the Josephson effects as well as magnetic flux conservation and are used not only for measuring biogmagnetic signals as induced for instance by the electrical currents within the human brain but also to read out cryogenic particle detectors, to explore mineral deposits within geoscience or for magnetic sensing at nanoscale level.
In this lecture, I will give an introduction into the fascinating field of Josephson junction based superconducting electronics, discuss different kinds of devices such as the well-known superconducting quantum interference device and highlight several applications for which superconducting electronics devices turn out to be a key technology.

\par
\newpage

\begin{center}
{{\large\bfseries High-Precision Tests of Quantum Electrodynamics - The g-Factor -}\par} \medskip

{\large Florian Köhler-Langes, Max-Planck-Institut für Astronomie, Heidelberg\par}
\end{center}

As the archetypal and best understood quantum field theory, quantum electrodynamics (QED) takes a prominent role in the highly successful Standard Model (SM) of physics. However, phenomena like dark matter and dark energy, which so far cannot be explained by the SM, challenge the model in a particular way. High-precision tests of QED might give a hint of physics beyond the SM. Furthermore, they lead to improved determinations of fundamental constants such as the electron mass or the fine-structure constant. In this lecture, we will introduce some high-precision tests of QED. A special focus will be set on measurements (and predictions) of the g-factor of the free electron, the free muon and the bound electron. In this context, a comprehensive introduction to state-of-the-art high-precision Penning-trap physics is given.\par
\newpage

\begin{center}
{{\large\bfseries Hot QCD Matter Produced in Ultra-Relativistic Heavy-Ion Collisions}\par} \medskip

{\large Silvia Masciocchi, GSI Darmstadt and Physikalisches Institut, Heidelberg University\par}
\end{center}

TBA

\noindent 
\par
\newpage

\begin{center}
{{\large\bfseries Radiation Biology: Physics at the Forefront of Cancer Therapy}\par} \medskip

{\large Joao Seco, German Cancer Research Center, Heidelberg\par}
\end{center}

Medical physics (also called biomedical physics, medical biophysics or applied physics in medicine) is, generally speaking, the application of physics concepts, theories and methods to medicine or healthcare. There are 4 main areas of medical physics specialty 1) radiation therapeutic physics, 2) medical imaging physics, 3) nuclear medicine physics and 4) health physics, which cover more than 90\% of all medical physics activities. Radiation therapeutic physicists work primarily in radiation oncology hospital departments, which specialize in cancer care. Radiation therapy (RT) is the most common treatment for cancer, being used in approximately 70\% of all cancers either alone or combined with surgery or chemotherapy. It uses high-energy particles or waves, such as x-rays, gamma rays, electron beams, protons, carbon ions, to kill or damage cancer cells.
\\
The first talk will give an overview of the physics behind radiation therapy. Starting from the first treatment of cancer done very early after the discovery of X-rays in 1895, to the technological developments in particle accelerators that permit modern day treatments with radiation and finally describing the impact of imaging, such as computed tomography (CT) and magnetic resonance imaging (MRI), in daily radiation therapy.
\\
The second talk will provide an overview of the biological effects of radiation, starting from the very small cellular level leading to the very large patient response. A brief introduction to cellular biology and radiation chemistry is given to allow a better understanding of how radiation effects occur. The talk will focus on explaining how radiation can damage cellular DNA under different environment and cellular conditions such as varying 1) oxygen levels, 2) dose rate of radiation delivery, 3) radiation particle type, 4) cell cycle, 5) cell death mechanism, etc. Radiation biology is at the heart of understanding the molecular mechanism of how radiation damages cells, allowing us to better treat cancer while minimizing side-effects.


\noindent 
\par
\newpage



{{\Large \noindent Special Invitation Lecture: }\par} \medskip

\begin{center}
{{\large\bfseries From snow to avalanches – a journey through scales and phase transitions }\par} \medskip

{\large Achille Capelli, WSL Institute for Snow and Avalanche Research SLF\par}
\end{center}

\noindent Snow on earth occurs in extraordinary variety of forms. It starts in the clouds with a phase transition from vapor to ice. As soon as the snowflakes reach the ground they sinter and form a continuous material constituted of an ice skeleton and air filling the pores. Subjected to the different weather conditions, the snow is transformed into different forms with various mechanical properties. This variability of mechanical properties of the snow cover allows the formation of avalanches. A dry-snow slab avalanche is released if a crack forms and propagates in a brittle weak layer below a cohesive slab. The initial crack is caused by a sudden surface load in the case of artificially triggered avalanches by e.g. a skier or an explosion. On the other hand, in the case of natural release the processes leading to initial crack formation are not fully understood. It is commonly assumed that a gradual damage process at the micro-scale leads to damage localization and the formation of the initial crack. This process can be seen as a phase transition. Moreover, the failure behavior of snow depends strongly on the imposed loading rate. Snow is brittle for fast loading and deforms plastically for slow loading. We will show how the acoustic emissions (AE) produced by microscopic cracking can be used to shed light on the damage process preceding failure and the mechanisms controlling the loading rate dependency. Moreover with a fiber bundle model (FBM) snow failure and the concurrent AE can be simulated. The FBM included two healing mechanisms that oppose the damage process: (a) sintering or regenerating broken fibers with time dependent probability, and (b) viscous deformation with resulting time dependent relaxation of load inhomogeneities. We show how the two healing mechanisms change the behavior of snow hindering the prediction of failure.\par
\newpage


