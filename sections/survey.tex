\section{Results of the Feedback Survey}
After the winterschool, a feedback survey was done to ask for the general impression of the participants of the winterschool and to get information on things that should be changed and on things that should remain. 34 persons answered the questionnaire. All of the answering persons agreed or strongly agreed on the statement "I enjoyed the winterschool". Almost all considered the winterschool to be well organized and that they felt well informed prior to the winterschool. The overnight-travelling was found to be not optimal by 6 persons, 5 were neutral and the remaining 23 preferred overnight traveling to traveling during the next day. 
\textbf{Hotel}
The rooms were found to be clean and comfortable, the food to be good and all dietary needs were met. There were no negative answers on the mentioned topics which gives a great feedback for the hotel itself. Only the combination of the overnight travel and the check-in at 3 p.m. was not optimal and we got several remarks on this. Nevertheless, this combination was cheaper than the during-the-day-travel and could make another spot for a participant.
\textbf{Lectures}
The students were content with the selection of topics (no negative answers) although more than 9 persons stated not having learned anything new in their own research topic but all agreed on having learned new aspects about other research fields within the HGSFP. For that reason, there was no wish for a deeper specialization of the lectures and it was found to be positive that the lectures were broad. We wanted to know, if a soft skill course would be interesting, but the response was ambivalent: 7 (strongly) disagreed, 10 remained neutral and 17 (strongly) agreed. Overall, the time per day that was spent on lectures was found to be sufficient.\\
The lectures with the most participants were \textit{Strongly Coupled Systems and the Applied Physics of Black Holes} (C. Ewerz), \textit{Building planets - a journey along 40 orders of magnitude} (T. Birnstiel) and \textit{Prospects and challenges of quantum computing} (M. Gaerttner).\\
The special lecture about avalanches and its corresponding excursion into the snow was a special event that was attended by 15-20 persons and the feedback was truly positive.\\
We asked, which other topics the students would have liked to learn something about, and the mentioned fields were: Soft skills, neural networks, meteorology, astrobiology, electroweak physics, computational physics, quantum many-body theory, atomic physics. 
This list could serve as an input for the future organization team to have an idea about which topics can be treated more into detail. 
\textbf{Poster Session}
The poster sessions started with elevator talks, which got positive feedback by all participants. The session itself was quite long (9 p.m. until open end) and was therefore considered to be long enough.
\textbf{Social Events}
The social event on the first evening (Eisstockschießen) was liked by almost everyone. It was also a good icebreaker to get into contact with the other students. On the last evening, a science quiz was offered, which was also found to be fun. Furthermore, 12/14 students liked the organized ski course and a comparable percentage liked the sledging event. We wanted to know if more social events should have been organized during the break, and the votes were: 5 agreed, 14 neutral and 14 disagreed.

In total, the feedback was very positive and we got the impression that the participants liked the winterschool a lot. There was a various offer of lectures and different possibilities of socializing and sports during the breaks, either organized by us or organized by the participants themselves. 

